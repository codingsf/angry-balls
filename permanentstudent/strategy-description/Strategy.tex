\documentclass[12pt,a4paper]{article}
\usepackage[utf8]{inputenc}
\usepackage[english,russian]{babel}
\usepackage{amssymb}
\usepackage{graphicx}
\usepackage{epstopdf}
\usepackage[english]{isodate}
\usepackage[noend]{algorithmic}
\usepackage{algorithm}
\usepackage[T2A]{fontenc}
\usepackage{url}
\usepackage{fancyref}
\usepackage[margin=0.5in]{geometry}
\usepackage{float}
\usepackage{tikz}
\usepackage{pgfplots}
\usepackage{caption}
\usepackage{multirow}
 
\begin{document}

\begin{center}
    \Large \bf Описание стратегии перемещения шара
\end{center}

Текущая стратегия перемещения шара основана на следующих рассуждениях и предположениях:
\begin{enumerate} 
\item Радиус стола достаточно велик для того, чтобы за 1 квант времени у каждого шара произошло не более одного соударения с краем стола.
\item Поскольку мы не знаем ходы других шаров, предсказывать соударение с ними очень сложно, поэтому будем каждый шар рассматривать независимо от других (как будто они могут проходить друг сквозь друга).
\item Допустим, у нас на столе есть несколько монет. Нам нужно оценить, до какой из них мы можем добраться быстрее всего и, если мы успеваем к ней раньше всех, двигаться к этой монете.
\item Для упрощения будем предполагать, что шар может забрать монету только в конце некоторого кванта времени.
\item Если не рассматривать столкновения между шарами, вероятно, оптимально будет при движении к монете на каждом шаге выбирать максимальное по модулю ускорение. Тогда оно полностью определяется углом, который образует вектор ускорения с положительным направлением оси \(Ox\). 
\item Пусть некоторый шар пришел из точки \(A\) в точку \(B\), выбрав последовательность ускорений \(\{(a_x^1,a_y^1),(a_x^2,a_y^2),...,(a_x^n,a_y^n)\}\). Тогда из непрерывности изменения углов можно предположить, что существует такое ускорение \((a_x^*,a_y^*)\), что, двигаясь каждый раз с таким ускорением, шар также придет в точку \(B\) (возможно, немного раньше или позже). Это предположение сильно упрощает оценку времени, необходимого шару для того, чтобы забрать монету.
\item Если радиусы шара и монеты не слишком малы по сравнению с радиусом стола, при оптимальном выборе ускорения шар пройдет достаточно небольшое расстояние (предположительно - не больше нескольких диаметров стола), прежде чем заберет монету, вне зависимости от расположения шара и монеты. Поэтому процесс движения шаров к монетам можно промоделировать, предварительно дискретизировав возможные ускорения. Например, можно рассматривать только ускорения вида 
\[\left(\cos \left( \frac{2 \pi k}{N}  \right), \sin \left( \frac{2 \pi k}{N}  \right) \right),\]
где \(0 \le k<N\)  -  целое; \(N\) - достаточно (но не слишком) большое число (например, \(N=10000\)).
\end{enumerate}

Сама стратегия состоит из следующих этапов:
\begin{enumerate} 
\item Для каждого из шаров оцениваем минимальное время, необходимое для того, чтобы добраться до одной из монет:
\begin{enumerate} 
\item Моделируем движение шара в течение, скажем, \(100\) квантов времени, для каждого из \(N\) ускорений (согласно предположению 6, мы считаем, что до конца пути будем двигаться именно с тем ускорением, которое мы выбрали в начале):
\begin{enumerate} 
\item Если шар в новой точке полностью попадает в пределы стола, все хорошо и идем дальше;
\item Если нет, то шар должен отразиться от стенки; бинарным поиском находим момент, когда он это делает, получаем точку, где это происходит, и отражаем шар, соответственно меняя его скорость.
\end{enumerate} 
\item Если в конце какого-то кванта при каком-то ускорении мы оказались в точке, откуда мы можем забрать монету, запоминаем ускорение, номер монеты и количество квантов времени, которое мы затратили, и прекращаем моделирование.
\end{enumerate} 
\item Теперь мы для каждого шара, в том числе и до нашего, знаем номер монеты, до которой ему ближе всего добираться, и оценку времени движения. Если мы успеваем добраться до монеты, которая нам ближе всего, раньше всех, возвращаем соответствующее ускорение, с которым нам это удастся сделать. Если же нет, либо монеты вообще отсутствуют, никуда не спешим и продолжаем двигаться, как двигались.
\end{enumerate} 

P.S. В принципе, с помощью моделирования можно получить минимальное время движения каждого шара до каждой монеты, а не только до ближайшей - ведь иногда выгоднее двигаться к монете, которая находится дальше, но до которой другим шарам добираться еще дольше. С другой стороны, это может значительно (в несколько раз) увеличить время работы стратегии, что может оказаться существенным.

\end{document}